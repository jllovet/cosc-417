\documentclass[12pt,letterpaper,final]{report}
\usepackage[utf8]{inputenc}
\usepackage{amsmath}
\usepackage{amsfonts}
\usepackage{amssymb}
\usepackage{amsthm}
\renewcommand\qedsymbol{$\blacksquare$}
\usepackage{enumerate}
\usepackage{hyperref}
\usepackage{pdfpages}
\usepackage{graphics}
\usepackage{graphicx}
\usepackage{tikz}
\usepackage{tikz-qtree}
\usetikzlibrary{automata,arrows}

\author{Marius Zimand}
\author{Jonathan Llovet}

\begin{document}

\fbox{
  \vbox{
    \begin{flushleft}
      Jonathan Llovet \\  % authors' names
      COSC 417 \\  %class
      2026-02-12, 11:59 PM (EST) \\  % date
    \end{flushleft}
    \center{\Large{\textbf{Assignment 1}}}
    %\end{mdframed}
  } % end vbox
} % end fbox
\vline

{\bf Problem 1.} Let $A = \{x,y\}$ and $B= \{x,y,z\}$.
\begin{enumerate}
  \item  Is $A$ a subset of $B$?

        Yes

  \item  Is $B$ a subset of $A$?

        No

  \item What is $A \cup B$?

        $\{x,y,z\}$

  \item  What is $A \cap B$?

        $\{x,y\}$

  \item  What is $A \times B$?

        $\{(x,x), (x,y), (x,z), (y,x), (y,y), (y,z)\}$

  \item  What is ${\cal P}(B)$? (${\cal P}(B)$ is the powerset of $B$).

        $\{
          \emptyset,
          \{x\},
          \{y\},
          \{z\},
          \{x,y\},
          \{x,z\},
          \{y,z\},
          \{x,y,z\}
          \}$

  \item What is ${\cal P}({\cal P}(A))$?

        Recall that
        \[
          \begin{array}{lll}
            A           & = \{x,y\}                              \\
            {\cal P}(A) & = \{\emptyset, \{x\}, \{y\}, \{x,y\}\} \\                                                                                                                                     \\ % quadruple
          \end{array}
        \]

        Arranging the elements by their cardinality, ${\cal P}({\cal P}(A))$ is as follows:
        \[
          \begin{array}{lll}
             & \Bigg\{ \emptyset,                                                                                                                                                               \\\\ % empty set
             & \{\emptyset\}, \{x\}, \{y\}, \{x, y\},                                                                                                                                           \\\\ % singletons
             & \Big\{\emptyset, \{x\}\Big\}, \Big\{\emptyset, \{y\}\Big\}, \Big\{\emptyset, \{x, y\}\Big\}, \Big\{\{x\}, \{y\}\Big\}, \Big\{\{x\}, \{x, y\}\Big\}, \Big\{\{y\}, \{x, y\}\Big\}, \\\\ % pairs
             & \Big\{\emptyset, \{x\}, \{y\}\Big\}, \Big\{\emptyset, \{x\}, \{x, y\}\Big\}, \Big\{\emptyset, \{y\}, \{x, y\}\Big\}, \Big\{\{x\}, \{y\}, \{x, y\}\Big\},                         \\\\ % triples
             & \Big\{\emptyset, \{x\}, \{y\}, \{x, y\}\Big\} \Bigg\}
          \end{array}
        \]

  \item What is the size of ${\cal P}(A) \times {\cal P}(B)$?


        The cardinality of a power set $X$ is $2^{\vert X \vert}$.

        $A = \{x,y\}$ and $B= \{x,y,z\}$. Their cardinalities are $\vert A \vert = 2$, $\vert B \vert = 3$.

        Hence, the size of ${\cal P}(A)$ is $2^2 = 4$, and the size of ${\cal P}(B)$ is $2^3 = 8$.

        Therefore the size of ${\cal P}(A) \times {\cal P}(B)$ is $2^{\vert A \vert} \times 2^{\vert B \vert}$, or

        $2^{2} \times 2^{3} = 4 \times 8 = 32$


\end{enumerate}

\pagebreak
{\bf Problem 2.}	Show that the set $A =\{-1, 0\} \cup \mathbb{N}$ is a countable infinite set by giving an explicit bijective function $f : {\mathbb N} \rightarrow A$.  Prove that your function $f$ is a bijection, by showing that it is 1-to-1 (no two natural numbers map to the same element) and onto (every element in $A$ is the image of some natural number).
\medskip

Let $f(n) = n - 2$.
First, we will write $A =\{-1, 0\} \cup \mathbb{N}$ as one set:

\[
  \begin{aligned}
    \mathbb{N} & = \{1,2,3,4,5,\ldots\}  \\
    A          & = \{-1,0,1,2,3,\ldots\}
  \end{aligned}
\]

By applying $f$ to each of the elements of $\mathbb{N}$ in turn, we will obtain:

\[
  \begin{aligned}
    f(\mathbb{N}) & = \{f(1), f(2), f(3), f(4), f(5), \ldots\} \\
                  & = \{-1,0,1,2,3,\ldots\}                    \\
                  & = A
  \end{aligned}
\]
We will prove that $f(n)$ is a bijection in two parts.
In the first part, we will show by contraposition that $f$ is injective, and in the second part we will show by induction that $f$ is surjective.

  {\bf Part 1: Injective (1-1)}: No two natural numbers map to the same element.

We wish to show that $f: \mathbb{N} \to A$ is injective, i.e. $\forall a, b \in \mathbb{N}, a \ne b \implies f(a) \ne f(b)$.
We will prove this by contraposition. The contrapositive of the preceding statement is $\forall a, b \in \mathbb{N}, f(a) = f(b) \implies a = b$.

Let
\[
  \begin{aligned}
    a & = c \\
    b & = d
  \end{aligned}
  \quad \text{where } c, d \in \mathbb{N}.
\]

Then
\[
  \begin{aligned}
     &            & f(c)  & = f(d)  &  & \text{using particular values for a, b} \\
     &            & c - 2 & = d - 2 &  & \text{by substitution for $f(n)$}       \\
     & \therefore & c     & = d     &  & \text{by adding 2 to both sides}
  \end{aligned}
\]

Therefore, because contrapositives are logically equivalent, we conclude that $\forall a, b \in \mathbb{N}, a \ne b \implies f(a) \ne f(b)$, and that the function $f$ is injective.

  {\bf Part 2: Surjective (Onto)}: Every element in A is the image of some natural number

TODO: Transcribe

\pagebreak
{\bf Problem 3.} For each of the following languages, give the state diagram of a DFA with the specified number of states that recognizes the language. The alphabet is $\Sigma = \{0,1\}$.

\begin{enumerate}

  \item $\{w \colon \mbox{ $w$ contains at least one $0$ and one $1$}\}$, with $4$ states.

  \item $\{w \colon \mbox{ $w$ has $0$ in every odd position}\}$, with $3$ states.


\end{enumerate}

\pagebreak
{\bf Problem 4.}  Recall that an infinite  set $A$ is countable if there is a \emph{bijective}  function $f: {\mathbb N} \rightarrow A$.  Show that if there is an \emph{onto} function
$g: {\mathbb N} \rightarrow A$, then $A$ is countable. In other words, you need to show how you can modify $g$ to obtain a bijective function $f$ that enumerates $A$.

\pagebreak
{\bf Problem 5. }
Let $A$ and $B$  be two infinite countable sets.  Show that  $A \cup  B$ is  also an infinite countable sets. You need to show how to obtain an enumeration of $A \cup B$ if you have an enumeration $A = \{f(1), f(2), \ldots \}$ and an enumeration of $B = \{g(1), g(2), \ldots \}$ (in other words you need to explain how you can list one-by-one all the elements  in $A \cup B$.)

\end{document}
