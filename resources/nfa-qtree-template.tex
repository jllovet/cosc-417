\documentclass[border=10pt]{standalone}
\usepackage{tikz}
\usepackage{tikz-qtree}
\usetikzlibrary{arrows.meta}

% ============================================================================
% NFA Computation Tree — tikz-qtree version
%
% Much less boilerplate than manual coordinate placement.
% The tree structure is expressed in a bracketed notation:
%   [.{parent} [.{child1} ] [.{child2} ] ]
%
% Dead branches (no valid transition) simply have no children.
%
% USAGE:
%   - Wrap state names in \st{name} for normal states
%   - Wrap in \ac{name} for accept (double-circle) states
%   - Use \edge[draw=none]; \node[draw=none]{}; for invisible phantom
%     nodes if you need spacing/alignment adjustments
%   - Compile with: pdflatex nfa_tree_qtree.tex
% ============================================================================

% --- Convenience macros for state nodes ---
\newcommand{\st}[1]{%
    \tikz[baseline=(n.base)]\node[state](n){$#1$};%
}
\newcommand{\ac}[1]{%
    \tikz[baseline=(n.base)]\node[accept state](n){$#1$};%
}

\begin{document}

\tikzset{
    % Node styles
    state/.style={
        circle, draw, minimum size=8mm, inner sep=0pt,
        font=\normalsize,
    },
    accept state/.style={
        state, double, double distance=1.5pt,
    },
    % Edge/arrow style — applied globally to the tree
    edge from parent/.style={
        draw, thick, ->, >=Stealth,
    },
    % Increase level distance (vertical spacing between levels)
    level distance=55pt,
    % Increase sibling distance (horizontal spacing)
    sibling distance=30pt,
}

\begin{tikzpicture}

% --- The tree itself ---
\Tree
[.\st{q_1}
    [.\st{q_1}
        [.\st{q_1}
            [.\st{q_1}
                [.\st{q_1}
                    [.\st{q_1}
                        \st{q_1}
                    ]
                    [.\st{q_2}
                        \st{q_3}
                    ]
                    [.\st{q_3} ]
                ]
                [.\st{q_2} ]
                [.\st{q_3}
                    [.\st{q_4}
                        [.\st{q_4}
                            \ac{q_4}
                        ]
                    ]
                ]
            ]
        ]
        [.\st{q_2}
            [.\st{q_3}
                [.\st{q_4}
                    [.\st{q_4}
                        \ac{q_4}
                    ]
                ]
            ]
        ]
        [.\st{q_3} ]
    ]
]

% --- Symbol labels on the left side ---
% We position these relative to the tree's bounding box.
% Adjust x-coordinates as needed for your tree width.
\begin{scope}[symbol label/.style={font=\large, anchor=east}]
    % Grab the root's y-coordinate as reference; levels are spaced by 55pt
    \path (current bounding box.north west) +(-0.5, 0) coordinate (labelanchor);

    \foreach \sym/\i in {0/1, 1/2, 0/3, 1/4, 1/5, 0/6} {
        \node[symbol label] at ([yshift={-(\i-0.5)*55pt}] labelanchor) {\sym};
    }
\end{scope}

% --- Caption ---
\node[font=\large\bfseries, anchor=north] at
    ([yshift=-15pt] current bounding box.south)
    {\textsc{Figure 1.29} \quad
     \normalfont The computation of $N_1$ on input 010110};

\end{tikzpicture}
\end{document}